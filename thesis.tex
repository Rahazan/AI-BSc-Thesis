\documentclass[12pt,fleqn,leqno,letterpaper]{article}

\include{preamble}

\title{Title}
\author{Guido Zuidhof\\
  \small{ todo }\\
  \small{Radboud Universiteit Nijmegen}\\
  \small{\texttt{guido.zuidhof@student.ru.nl}}
}
\date{November 18, 2014}

\begin{document}

%\setstretch{1.00}
\maketitle

% -- Table of Contents --
% (would go here)

% -- set document spacing --
% \setstretch{1.09}  % single line
% \setstretch{1.30}  % single wide-spaced
% \setstretch{1.50}  % one and a half spacing

\begin{abstract}
\end{abstract}

\section{Introduction}
\subsection{Cognitive load}

\subsection{Physiological response to cognitive load}
How a human body responds to cognitive load/stress. 

\subsection{Determining cognitive load}
How this cognitive load could be determined, i.e. by measurement.

\subsubsection{Measurement of physiological response}
Measuring heart rate, breathing, eeg, gsr, etc.

\subsubsection{Classification}
How sense can be made from this data

\subsubsection{Applications}
How this can be used.

\subsection{Research aim}
Using features extracted from GSR data to determine cognitive load, data gathered from experiment.

\section{Methods}

\subsection{Experiment}
The experiment is mostly based on the first experiment conducted by Nourbakhsh et al., \_meer uitleg todo\_
\subsubsection{Apparatus}
The task was displayed in a 15.6" laptop screen, input was given using a standard computer mouse.
Two galvanic skin response sensors were used in this experiment. The participants wore both at the same time on the left arm. Both were set to a sampling frequency of 12Hz. 

\subsubsection{Affectiva Q Sensor}
This sensor is embedded in a wrist band, it uses a dry electrode. The sensor side of the wrist was placed on the bottom of the wrist.

\subsubsection{BIOPAC MP30}
The BIOPAC MP30 was used in conjunction with finger GSR sensors. This sensor wraps around two fingers, in this experiment the index and ring finger. It uses gel electrodes, Grass EC33 Electrode Paste was used.

\subsubsection{Task}

\subsubsection{Experimental design}
The task consisted of 12 arithmetic tasks, with 4 difficulty levels. Three of every difficulty level were completed by the participants, in a random order. 

The task consisted of adding up three numbers, and selecting the right answer from three options. The numbers to add were showed one by one, for four seconds. Before a task starts, a number of stars is shown equal to the difficulty level, for eight seconds. The tasks followed eachother without breaks. There was no time limit for selecting the answer, there was no feedback whether the selected answer was correct.


\subsubsection{Procedure}
The participants were first explained the task and seated in front of the laptop. Then the sensors were applied to the left hand and wrist. The participant is then instructed to place the left hand on the table in front of them and not to move it, in order to prevent movement or motion artifacts.

\subsection{Analysis}
\subsubsection{Preprocessing}
\subsubsection{Transformation}
\subsubsection{Data mining}
\subsubsection{Interpretation}

\section{Results}
\subsection{..?}
\subsection{..?}

\section{Discussion}
\subsection{Viability}
\subsection{Future research}
\section{Conclusion}





% -- Bibliography (APA style)
\bibliography{references}

\end{document}
